%===============================================================================
% COVID-19 Reproduction Numbers in South Africa
% Kevin Durant
% May 2020
%===============================================================================

\documentclass[12pt,a4paper]{article}

% \usepackage{booktabs} % better looking tables.
\usepackage{mathtools} % also loads amsmath.
\usepackage{microtype}
\usepackage{iftex}
\ifPDFTeX
  \usepackage[T1]{fontenc}
  \usepackage[utf8]{inputenc}
  \usepackage{amssymb}
\else
  % Unicode-math should be loaded after other maths- or font-related packages.
  % It loads amsmath and fontspec if necessary, and enables a custom version of
  % the Latin Modern Math font by default.
  \usepackage{unicode-math}
%   \setmainfont{STIX Two Text}
%   \setmathfont{STIX Two Math}
\fi
\usepackage{biblatex}
\usepackage{svg}
\usepackage{hyperref}
\addbibresource{cvza.bib}

% General maths commands %======================================================

\DeclarePairedDelimiter\lr{\lparen}{\rparen}  % sized parentheses.
\DeclareMathOperator\Pb{P}                    % probability.
\DeclareMathOperator\B{B}                     % beta distribution.
\DeclareMathOperator\BP{BP}                   % beta prime distribution.
\DeclareMathOperator\NB{NB}                   % negative binomial distribution.

% Title %=======================================================================

\title{Model Summary}
\author{Kevin Durant}
\date{}

% Document %====================================================================

\begin{document}

\maketitle

\section{Definitions} %=========================================================

$k_t$: number of new infections on day $t$. \\
$\lambda_t$: underlying daily rate of infection (and expected number of new
infections) on day $t$.

\section{Assumptions} %=========================================================

Firstly, that $k_t$ depends on $\lambda_t$ via a negative binomial distribution
with dispersion parameter $r$:
\begin{equation*}
  \Pb(k_t \mid r, \lambda_t)
  = \NB\lr*{k_t \Bigm\vert r, \frac{\lambda}{r + \lambda} = p_t}.
\end{equation*}

Secondly, that the posterior on $p_t$ is a beta distribution (which is the
conjugate prior for the negative binomial with known dispersion):
\begin{align*}
  \Pb(p_t \mid k_1, \dots, k_t) &= \B(p_t \mid \alpha_t, \beta_t) \\
  \Leftrightarrow \Pb\lr*{\frac{\lambda_t}{r} \Bigm\vert k_1, \dots, k_t}
    &= \BP\lr*{\frac{\lambda_t}{r} \Bigm\vert \alpha_t, \beta_t}.
\end{align*}
As indicated here, this is equivalent to the assumption of a beta prime prior
on $\lambda_t/r$.

Thirdly, we assume that the predictive prior on $p_t$ is related to the
posterior on $p_{t-1}$ as follows:
\begin{align*}
  \Pb(p_t \mid k_1, \dots, k_{t-1})
    &= \B\lr*{p_t \Bigm\vert \frac{\alpha_{t-1}}{c}, \frac{\beta_{t-1}}{c}} \\
  \Leftrightarrow \Pb\lr*{\frac{\lambda_t}{r} \Bigm\vert k_1, \dots, k_{t-1}}
    &= \BP\lr*{\frac{\lambda_t}{r} \Bigm\vert
    \frac{\alpha_{t-1}}{c}, \frac{\beta_{t-1}}{c}}.
\end{align*}

\section{Results} %=============================================================

The parameters of the posterior distributions satisfy
\begin{align*}
  \alpha_t &= \frac{\alpha_{t-1}}{c} + k_t
    = \frac{a_1}{c^{t-1}} + \sum_{i=0}^{t-1} \frac{k_{t-i}}{c^i}, \\
  \beta_t &= \frac{\beta_{t-1}}{c} + r
    = \frac{b_1}{c^{t-1}} + \sum_{i=0}^{t-1} \frac{r}{c_i},
\end{align*}
and the marginal likelihood of the model parameters $r$ and $c$ is
\begin{align*}
  \Pb(k_1, \dots, k_t \mid r, c)
    &= \prod_{t=1}^{t} \Pb(k_t \mid k_1, \dots, k_{t-1}, r, c) \\
  &= \prod_{t=1}^{t} \frac{1}{k_t\B(k_t, r)}
    \frac{\B(a_t + k_t, b_t + r)}{\B(a_t, b_t)} \quad (\text{if } k_t > 0).
\end{align*}

\end{document}